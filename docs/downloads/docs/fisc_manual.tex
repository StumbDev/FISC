\documentclass[a4paper,11pt]{book}
\usepackage[utf8]{inputenc}
\usepackage{graphicx}
\usepackage{listings}
\usepackage{color}
\usepackage{hyperref}
\usepackage{fancyhdr}
\usepackage{float}
\usepackage{tikz}
\usepackage{geometry}

% Define colors for code listings
\definecolor{codegreen}{rgb}{0,0.6,0}
\definecolor{codegray}{rgb}{0.5,0.5,0.5}
\definecolor{codepurple}{rgb}{0.58,0,0.82}
\definecolor{backcolour}{rgb}{0.95,0.95,0.92}

% Configure code listings
\lstdefinestyle{fxmlstyle}{
    backgroundcolor=\color{backcolour},   
    commentstyle=\color{codegreen},
    keywordstyle=\color{codepurple},
    numberstyle=\tiny\color{codegray},
    stringstyle=\color{codegreen},
    basicstyle=\ttfamily\footnotesize,
    breakatwhitespace=false,         
    breaklines=true,                 
    captionpos=b,                    
    keepspaces=true,                 
    numbers=left,                    
    numbersep=5pt,                  
    showspaces=false,                
    showstringspaces=false,
    showtabs=false,                  
    tabsize=2
}

\title{FISC-V User Manual}
\author{FISC-V Development Team}
\date{\today}

\begin{document}

\frontmatter
\maketitle

\tableofcontents

\mainmatter

\chapter{Introduction}
\section{Overview}
FISC-V is a highly configurable virtual processing unit that supports multiple architectures, including RISC-V variants and the x86 family. This manual provides comprehensive documentation for setting up and using FISC-V.

\section{Features}
\begin{itemize}
    \item Multiple architecture support (RISC-V, x86, 8-bit, 4-bit)
    \item Configurable memory and cache systems
    \item Real and protected mode support for x86
    \item Integrated debugging capabilities
    \item Multiple user interfaces (GUI, TUI, CLI)
\end{itemize}

\chapter{Installation}
\section{System Requirements}
\begin{itemize}
    \item 64-bit operating system (Windows, Linux, or macOS)
    \item 2GB RAM minimum
    \item 100MB disk space
\end{itemize}

\section{Building from Source}
\begin{lstlisting}[language=bash]
mkdir build
cd build
cmake ..
make
\end{lstlisting}

\chapter{Configuration}
\section{FXml Format}
FXml is the configuration format used by FISC-V. Here's an example configuration:

\begin{lstlisting}[style=fxmlstyle]
ARCHITECTURE = 80386DX
ADDRESS_BUS_WIDTH = 32
DATA_BUS_WIDTH = 32
MEMORY_SIZE = 16777216
START_ADDRESS = 0x0000
CPU_FREQUENCY = 33000000
\end{lstlisting}

\section{Configuration Parameters}
\subsection{Basic Parameters}
\begin{itemize}
    \item \texttt{ARCHITECTURE}: CPU architecture type
    \item \texttt{MEMORY\_SIZE}: Total memory size in bytes
    \item \texttt{START\_ADDRESS}: Program start address
\end{itemize}

\subsection{Architecture-Specific Parameters}
\begin{itemize}
    \item \texttt{X86\_REAL\_MODE}: Enable x86 real mode
    \item \texttt{X86\_PROTECTED\_MODE}: Enable x86 protected mode
    \item \texttt{SEGMENT\_REGISTERS}: Enable segmentation
\end{itemize}

\chapter{Using FISC-V}
\section{Command Line Interface}
The CLI provides basic configuration management:
\begin{lstlisting}[language=bash]
$ ./fisc_cli
load <filename>    - Load configuration
edit <param> <val> - Edit parameter
save <filename>    - Save configuration
run               - Start VPU
\end{lstlisting}

\section{Text User Interface}
The TUI provides an ncurses-based interface with menu navigation:
\begin{itemize}
    \item Use arrow keys to navigate
    \item Enter to select options
    \item ESC to go back
\end{itemize}

\section{Graphical User Interface}
The GUI provides a modern interface with:
\begin{itemize}
    \item Configuration file management
    \item Real-time parameter editing
    \item Visual status indicators
\end{itemize}

\chapter{Architecture Support}
\section{x86 Family}
\subsection{386DX Configuration}
\begin{lstlisting}[style=fxmlstyle]
ARCHITECTURE = 80386DX
ADDRESS_BUS_WIDTH = 32
DATA_BUS_WIDTH = 32
MEMORY_SIZE = 16777216
FPU_TYPE = 80387
MMU_TYPE = paging_and_segmentation
\end{lstlisting}

\subsection{486DX2 Configuration}
\begin{lstlisting}[style=fxmlstyle]
ARCHITECTURE = 80486DX2
ADDRESS_BUS_WIDTH = 32
DATA_BUS_WIDTH = 32
MEMORY_SIZE = 67108864
FPU_TYPE = internal
CLOCK_MULTIPLIER = 2
\end{lstlisting}

\section{RISC-V Support}
\subsection{RV32I Configuration}
\begin{lstlisting}[style=fxmlstyle]
ARCHITECTURE = RISC-V-32
ADDRESS_BUS_WIDTH = 32
DATA_BUS_WIDTH = 32
INSTRUCTION_SET_EXTENSIONS = base
\end{lstlisting}

\chapter{Debugging}
\section{Debug Levels}
\begin{itemize}
    \item \texttt{none}: No debug output
    \item \texttt{minimal}: Basic status messages
    \item \texttt{normal}: Standard debugging info
    \item \texttt{verbose}: Detailed execution trace
    \item \texttt{debug}: Full system state
\end{itemize}

\section{Instruction Tracing}
Enable instruction tracing with:
\begin{lstlisting}[style=fxmlstyle]
DEBUG_LEVEL = verbose
TRACE_INSTRUCTIONS = true
\end{lstlisting}

\chapter{Advanced Topics}
\section{Memory Management}
\subsection{Paging Configuration}
\begin{itemize}
    \item Page size configuration
    \item Page table structure
    \item TLB settings
\end{itemize}

\section{Cache Configuration}
\begin{itemize}
    \item L1 cache settings
    \item Cache coherency
    \item Write policies
\end{itemize}

\chapter{Architecture-Specific Features}
\section{x86 Family}
\subsection{Real Mode Operation}
Real mode provides compatibility with 8086/8088 software and operates with:
\begin{itemize}
    \item 20-bit segmented addressing
    \item 1MB address space
    \item No memory protection
    \item Direct hardware access
\end{itemize}

\subsection{Protected Mode}
Protected mode enables advanced features:
\begin{itemize}
    \item Memory protection
    \item Virtual memory support
    \item Task switching
    \item Privilege levels (rings 0-3)
\end{itemize}

\subsection{Virtual 8086 Mode}
Available on 386 and later:
\begin{itemize}
    \item Multiple virtual 8086 machines
    \item Hardware virtualization
    \item DOS compatibility
\end{itemize}

\section{RISC-V Implementation}
\subsection{Base Integer Instructions (RV32I)}
Core instruction set includes:
\begin{itemize}
    \item Arithmetic operations
    \item Logical operations
    \item Load/Store operations
    \item Control transfer
\end{itemize}

\subsection{Standard Extensions}
Supported extensions:
\begin{itemize}
    \item M - Integer Multiplication and Division
    \item A - Atomic Instructions
    \item F - Single-Precision Floating-Point
    \item D - Double-Precision Floating-Point
\end{itemize}

\chapter{Performance Optimization}
\section{Clock Configuration}
\subsection{Clock Multiplier Settings}
\begin{itemize}
    \item Base clock selection
    \item Multiplier ranges
    \item Timing constraints
\end{itemize}

\subsection{Wait States}
Memory timing configuration:
\begin{itemize}
    \item DRAM refresh timing
    \item Memory access latency
    \item I/O wait states
\end{itemize}

\section{Cache Optimization}
\subsection{Cache Policies}
Available cache policies:
\begin{itemize}
    \item Write-through
    \item Write-back
    \item Write-allocate
    \item No-write-allocate
\end{itemize}

\subsection{Cache Organization}
Cache structure options:
\begin{itemize}
    \item Direct mapped
    \item Set associative
    \item Fully associative
\end{itemize}

\chapter{Development Tools}
\section{Debugging Features}
\subsection{Instruction Tracing}
Trace configuration options:
\begin{itemize}
    \item Instruction execution
    \item Memory access
    \item I/O operations
    \item Interrupt handling
\end{itemize}

\subsection{Breakpoints}
Types of breakpoints:
\begin{itemize}
    \item Execution breakpoints
    \item Memory access breakpoints
    \item I/O breakpoints
    \item Conditional breakpoints
\end{itemize}

\section{Performance Monitoring}
\subsection{Hardware Counters}
Available counters:
\begin{itemize}
    \item Instruction counts
    \item Cache hits/misses
    \item Branch prediction
    \item Memory access patterns
\end{itemize}

\subsection{Profiling Tools}
Built-in profiling features:
\begin{itemize}
    \item Execution time analysis
    \item Memory usage patterns
    \item I/O bottleneck detection
    \item Cache efficiency analysis
\end{itemize}

\chapter{System Integration}
\section{I/O Subsystem}
\subsection{Port Configuration}
I/O port settings:
\begin{itemize}
    \item Port address ranges
    \item I/O permissions
    \item Port timing
    \item Interrupt configuration
\end{itemize}

\subsection{DMA Support}
DMA features:
\begin{itemize}
    \item Channel configuration
    \item Transfer modes
    \item Priority levels
    \item Buffer management
\end{itemize}

\section{Interrupt Handling}
\subsection{Interrupt Controllers}
Supported controllers:
\begin{itemize}
    \item 8259 PIC emulation
    \item APIC support
    \item Custom interrupt routing
\end{itemize}

\subsection{Interrupt Modes}
Available modes:
\begin{itemize}
    \item Real mode interrupts
    \item Protected mode interrupts
    \item Virtual 8086 mode interrupts
    \item Message-based interrupts
\end{itemize}

\chapter{Configuration Examples}
\section{Common Scenarios}
\subsection{DOS Gaming System}
\begin{lstlisting}[style=fxmlstyle]
ARCHITECTURE = 80486DX2
ADDRESS_BUS_WIDTH = 32
DATA_BUS_WIDTH = 32
MEMORY_SIZE = 16777216
X86_REAL_MODE = true
FPU_TYPE = internal
CLOCK_MULTIPLIER = 2
\end{lstlisting}

\subsection{Unix Workstation}
\begin{lstlisting}[style=fxmlstyle]
ARCHITECTURE = 80486DX
ADDRESS_BUS_WIDTH = 32
DATA_BUS_WIDTH = 32
MEMORY_SIZE = 67108864
X86_PROTECTED_MODE = true
MMU_TYPE = paging_and_segmentation
CLOCK_MULTIPLIER = 1
\end{lstlisting}

\subsection{Embedded Controller}
\begin{lstlisting}[style=fxmlstyle]
ARCHITECTURE = RISC-V-32
ADDRESS_BUS_WIDTH = 32
DATA_BUS_WIDTH = 32
MEMORY_SIZE = 1048576
INSTRUCTION_SET_EXTENSIONS = base,m
DEBUG_LEVEL = minimal
\end{lstlisting}

\chapter{Troubleshooting}
\section{Common Issues}
\subsection{Boot Problems}
Common boot issues and solutions:
\begin{itemize}
    \item BIOS initialization failures
    \item Memory configuration errors
    \item CPU mode switching problems
\end{itemize}

\subsection{Performance Issues}
Performance troubleshooting:
\begin{itemize}
    \item Cache configuration problems
    \item Memory timing issues
    \item Clock synchronization errors
\end{itemize}

\appendix
\chapter{Error Messages}
\section{System Messages}
\begin{description}
    \item[E001] Invalid configuration parameter
    \item[E002] Memory allocation failed
    \item[E003] Invalid instruction execution
    \item[E004] Cache coherency error
\end{description}

\section{Configuration Errors}
\begin{description}
    \item[C001] Invalid architecture specification
    \item[C002] Incompatible mode settings
    \item[C003] Invalid memory size
    \item[C004] Unsupported feature combination
\end{description}

\chapter{Configuration Reference}
\section{Complete Parameter List}
Comprehensive listing of all configuration parameters with:
\begin{itemize}
    \item Parameter name and type
    \item Valid values and constraints
    \item Default settings
    \item Dependencies and conflicts
\end{itemize}

\backmatter
\chapter{Glossary}
\begin{description}
    \item[VPU] Virtual Processing Unit
    \item[FXml] FISC Configuration Markup Language
    \item[MMU] Memory Management Unit
    \item[FPU] Floating Point Unit
    \item[TLB] Translation Lookaside Buffer
    \item[SIMD] Single Instruction, Multiple Data
    \item[DMA] Direct Memory Access
    \item[APIC] Advanced Programmable Interrupt Controller
\end{description}

\chapter{Bibliography}
\begin{thebibliography}{9}
\bibitem{intel386} Intel Corporation. \emph{i386 Programmer's Reference Manual}. 1986.
\bibitem{intel486} Intel Corporation. \emph{i486 Programmer's Reference Manual}. 1990.
\bibitem{riscv} RISC-V Foundation. \emph{The RISC-V Instruction Set Manual}. 2019.
\end{thebibliography}

\end{document} 